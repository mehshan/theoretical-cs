\documentclass[11pt]{article}


\usepackage{fullpage}
\usepackage{graphicx}
\usepackage{amsmath}
\usepackage{amssymb}
\usepackage{amsthm}
\usepackage{fancyvrb}

\newcommand{\myname}{Mehshan Mustafa}

\newenvironment{theorem}[2][Theorem]{\begin{trivlist}
\item[\hskip \labelsep {\bfseries #1}\hskip \labelsep {\bfseries #2.}]}{\end{trivlist}}
\newenvironment{lemma}[2][Lemma]{\begin{trivlist}
\item[\hskip \labelsep {\bfseries #1}\hskip \labelsep {\bfseries #2.}]}{\end{trivlist}}
\newenvironment{exercise}[2][Exercise]{\begin{trivlist}
\item[\hskip \labelsep {\bfseries #1}\hskip \labelsep {\bfseries #2.}]}{\end{trivlist}}
\newenvironment{problem}[2][Problem]{\begin{trivlist}
\item[\hskip \labelsep {\bfseries #1}\hskip \labelsep {\bfseries #2.}]}{\end{trivlist}}
\newenvironment{question}[2][Question]{\begin{trivlist}
\item[\hskip \labelsep {\bfseries #1}\hskip \labelsep {\bfseries #2.}]}{\end{trivlist}}
\newenvironment{corollary}[2][Corollary]{\begin{trivlist}
\item[\hskip \labelsep {\bfseries #1}\hskip \labelsep {\bfseries #2.}]}{\end{trivlist}}
\newenvironment{solution}{\begin{proof}[Solution]}{\end{proof}}
\newenvironment{idea}[2][Proof Idea.]{\textit{#1} #2}



\parindent0in
\pagestyle{plain}
\thispagestyle{plain}


\usepackage{csquotes}

\newcommand{\dated}{\today}
\newcommand{\token}[1]{\langle \text{#1} \rangle}

\begin{document}

\textbf{Introduction to the Theory of
Computation}\hfill\textbf{\myname}\\[0.01in]
\textbf{Chapter 4: Decidability}\hfill\textbf{\dated}\\
\smallskip\hrule\bigskip

\begin{problem}{4.22}
Let $PREFIX-FREE_{REX} = \{\langle R \rangle \ | \ R \text{ is a regular expression and } L(R) \text{ is prefix-free}\}$. Show that $PREFIX-FREE{REX}$ is decidable. Why does a similar approach fail to show that $PREFIX-FREE_{ CFG}$ is decidable?
\end{problem}

\begin{proof}
We present a \textbf{TM} $I$ that decides $PREFIX-FREE_{REX}$.  \\
\\
$I =$ \textquotedblleft On input $\langle R \rangle$, where $R$ is a regular expression:
\begin{enumerate}
\item Convert $R$ to equivalent NFA $N$.
\item Construct an NFA $M$, such that $L(M) = NOPREFIX(N)$ by following the construction given in solution to Problem 1.40 a.
\item Convert $M$ to equivalent DFA $D$.
\item Test $L(D) = \phi$ using the $E_{DFA}$ decider $T$ from Theorem 4.4.
\item If $T$ accepts, \textit{accept}; if $F$ rejects, \textit{reject}.\textquotedblright
\end{enumerate}
\end{proof}
The class of context-free languages is not closed under $NOPREFIX$\footnote{Refer to the proof given in solution to Problem 2.41 a.}. Therefore, the approach used to construct a decider for $PREFIX-FREE_{REX}$ cannot be used to show that $PREFIX-FREE_{ CFG}$ is decidable.
\end{document}