\documentclass[11pt]{article}


\usepackage{fullpage}
\usepackage{graphicx}
\usepackage{amsmath}
\usepackage{amssymb}
\usepackage{amsthm}
\usepackage{fancyvrb}

\newcommand{\myname}{Mehshan Mustafa}

\newenvironment{theorem}[2][Theorem]{\begin{trivlist}
\item[\hskip \labelsep {\bfseries #1}\hskip \labelsep {\bfseries #2.}]}{\end{trivlist}}
\newenvironment{lemma}[2][Lemma]{\begin{trivlist}
\item[\hskip \labelsep {\bfseries #1}\hskip \labelsep {\bfseries #2.}]}{\end{trivlist}}
\newenvironment{exercise}[2][Exercise]{\begin{trivlist}
\item[\hskip \labelsep {\bfseries #1}\hskip \labelsep {\bfseries #2.}]}{\end{trivlist}}
\newenvironment{problem}[2][Problem]{\begin{trivlist}
\item[\hskip \labelsep {\bfseries #1}\hskip \labelsep {\bfseries #2.}]}{\end{trivlist}}
\newenvironment{question}[2][Question]{\begin{trivlist}
\item[\hskip \labelsep {\bfseries #1}\hskip \labelsep {\bfseries #2.}]}{\end{trivlist}}
\newenvironment{corollary}[2][Corollary]{\begin{trivlist}
\item[\hskip \labelsep {\bfseries #1}\hskip \labelsep {\bfseries #2.}]}{\end{trivlist}}
\newenvironment{solution}{\begin{proof}[Solution]}{\end{proof}}
\newenvironment{idea}[2][Proof Idea.]{\textit{#1} #2}



\parindent0in
\pagestyle{plain}
\thispagestyle{plain}

\usepackage{csquotes}
\usepackage[shortlabels]{enumitem}

\newcommand{\dated}{\today}
\newcommand{\token}[1]{\langle \text{#1} \rangle}

\begin{document}

\textbf{Introduction to the Theory of
Computation}\hfill\textbf{\myname}\\[0.01in]
\textbf{Chapter 7: Time Complexity}\hfill\textbf{\dated}\\
\smallskip\hrule\bigskip

\begin{problem}{7.18}
Show that if $P = NP$, then every language $A \in P$, except $A = \emptyset$ and $A = \Sigma^{*}$, is NP-complete.
\end{problem}

\begin{proof}Let $A$ be any language in $P$, except $\emptyset$ and $\Sigma^{*}$. To show that if $P = NP$, then $A$ is NP-complete, we show that $SAT \leq_p A$. Let $S$ be the TM that decides $SAT$ in polynomial time. There exist two strings $w_a$ and $w_b$, such that $w_a \in A$ and $w_b \notin A$. Construct polynomial time reduction $F$ from $SAT$ to $A$ as follows: \\

$F =$ \textquotedblleft On input $\langle \phi \rangle$, where $\phi$ is a Boolean formula:
\begin{enumerate}
\item Run $S$ on $\langle \phi \rangle$.
\item If $S$ accepts, then output $w_a$, otherwise output $w_b$.\textquotedblright
\end{enumerate}
\end{proof}
\end{document}