\documentclass[11pt]{article}


\usepackage{fullpage}
\usepackage{graphicx}
\usepackage{amsmath}
\usepackage{amssymb}
\usepackage{amsthm}
\usepackage{fancyvrb}

\newcommand{\myname}{Mehshan Mustafa}

\newenvironment{theorem}[2][Theorem]{\begin{trivlist}
\item[\hskip \labelsep {\bfseries #1}\hskip \labelsep {\bfseries #2.}]}{\end{trivlist}}
\newenvironment{lemma}[2][Lemma]{\begin{trivlist}
\item[\hskip \labelsep {\bfseries #1}\hskip \labelsep {\bfseries #2.}]}{\end{trivlist}}
\newenvironment{exercise}[2][Exercise]{\begin{trivlist}
\item[\hskip \labelsep {\bfseries #1}\hskip \labelsep {\bfseries #2.}]}{\end{trivlist}}
\newenvironment{problem}[2][Problem]{\begin{trivlist}
\item[\hskip \labelsep {\bfseries #1}\hskip \labelsep {\bfseries #2.}]}{\end{trivlist}}
\newenvironment{question}[2][Question]{\begin{trivlist}
\item[\hskip \labelsep {\bfseries #1}\hskip \labelsep {\bfseries #2.}]}{\end{trivlist}}
\newenvironment{corollary}[2][Corollary]{\begin{trivlist}
\item[\hskip \labelsep {\bfseries #1}\hskip \labelsep {\bfseries #2.}]}{\end{trivlist}}
\newenvironment{solution}{\begin{proof}[Solution]}{\end{proof}}
\newenvironment{idea}[2][Proof Idea.]{\textit{#1} #2}



\parindent0in
\pagestyle{plain}
\thispagestyle{plain}

\usepackage{csquotes}
\usepackage[shortlabels]{enumitem}

\newcommand{\dated}{\today}
\newcommand{\token}[1]{\langle \text{#1} \rangle}

\begin{document}

\textbf{Introduction to the Theory of
Computation}\hfill\textbf{\myname}\\[0.01in]
\textbf{Chapter 5: Reducibility}\hfill\textbf{\dated}\\
\smallskip\hrule\bigskip

\begin{problem}{5.18}
Show that the Post Correspondence Problem is undecidable over the binary alphabet $\Sigma = {0,1}$.
\end{problem}

\begin{proof}
Let $PCP_b = \{\langle P \rangle \ | \ P \text{ is an instance of } PCP \text{ over binary alphabet } \Sigma = \{0, 1\}\}$. To show that $PCP_b$ is undecidable, we give a reduction from $PCP$ to $PCP_b$. Let $\Sigma_{PCP}$ be a finite set of symbols $a_1, a_2, \dots, a_n$, where $n \geq 1$. Let $g$ be a function that maps each symbol $a_i$ to the binary representation of its index $i$ using $\lfloor log_{2}n \rfloor + 1$ characters. A reduction $f$ from $PCP$ to $PCP_b$ can be constructed using $g$ that converts an instance of $PCP$ to $PCP_b$. As $PCP$ is undecidable and $PCP \leq_{m} PCP_b$, so according to the Corollary 5.23, $PCP_b$ is also undecidable.
\end{proof}
\end{document}