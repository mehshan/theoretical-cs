\documentclass[11pt]{article}


\usepackage{fullpage}
\usepackage{graphicx}
\usepackage{amsmath}
\usepackage{amssymb}
\usepackage{amsthm}
\usepackage{fancyvrb}

\newcommand{\myname}{Mehshan Mustafa}

\newenvironment{theorem}[2][Theorem]{\begin{trivlist}
\item[\hskip \labelsep {\bfseries #1}\hskip \labelsep {\bfseries #2.}]}{\end{trivlist}}
\newenvironment{lemma}[2][Lemma]{\begin{trivlist}
\item[\hskip \labelsep {\bfseries #1}\hskip \labelsep {\bfseries #2.}]}{\end{trivlist}}
\newenvironment{exercise}[2][Exercise]{\begin{trivlist}
\item[\hskip \labelsep {\bfseries #1}\hskip \labelsep {\bfseries #2.}]}{\end{trivlist}}
\newenvironment{problem}[2][Problem]{\begin{trivlist}
\item[\hskip \labelsep {\bfseries #1}\hskip \labelsep {\bfseries #2.}]}{\end{trivlist}}
\newenvironment{question}[2][Question]{\begin{trivlist}
\item[\hskip \labelsep {\bfseries #1}\hskip \labelsep {\bfseries #2.}]}{\end{trivlist}}
\newenvironment{corollary}[2][Corollary]{\begin{trivlist}
\item[\hskip \labelsep {\bfseries #1}\hskip \labelsep {\bfseries #2.}]}{\end{trivlist}}
\newenvironment{solution}{\begin{proof}[Solution]}{\end{proof}}
\newenvironment{idea}[2][Proof Idea.]{\textit{#1} #2}



\parindent0in
\pagestyle{plain}
\thispagestyle{plain}

\usepackage{csquotes}
\usepackage[shortlabels]{enumitem}

\newcommand{\dated}{\today}
\newcommand{\token}[1]{\langle \text{#1} \rangle}

\begin{document}

\textbf{Introduction to the Theory of
Computation}\hfill\textbf{\myname}\\[0.01in]
\textbf{Chapter 7: Time Complexity}\hfill\textbf{\dated}\\
\smallskip\hrule\bigskip

\begin{problem}{7.32}
This problem is inspired by the single-player game \textit{Minesweeper}, generalized to an arbitrary graph. Let $G$ be an undirected graph, where each node either contains a single, hidden \textit{mine} or is empty. The player chooses nodes, one by one. If the player chooses a node containing a mine, the player loses. If the player chooses an empty node, the player learns the number of neighboring nodes containing mines. (A neighboring node is one connected to the chosen node by an edge.) The player wins if and when all empty nodes have been so chosen.\\

In the \textit{mine consistency problem}, you are given a graph $G$ along with numbers labeling some of $G$'s nodes. You must determine whether a placement of mines on the remaining nodes is possible, so that any node $v$ that is labeled $m$ has exactly $m$ neighboring nodes containing mines. Formulate this problem as a language and show that it is NP-complete.
\end{problem}

\begin{proof}
Let $MINE-CONSISTENCY = \{\langle G \rangle \ | \ G \text{ is an undirected graph along with numbers} \\ \text{labeling some of } G\text{'s nodes, and a placement of mines on the remaining nodes is possible, so that} \\ \text{any node } v \text{ that is labeled } m \text{ has exactly } m \text{ neighboring nodes containing mines.}\}$. \\

To show that $MINE-CONSISTENCY$ is NP-complete, we must show that it is in NP and that all NP-problems are polynomial time reducible to it. The first part is easy; a certificate is simply the placement of mines. To prove the second part, we show that $3COLOR$ is polynomial time reducible to $MINE-CONSISTENCY$. The reduction converts an undirected graph $G_1$ into undirected graph $G_2$ along with numbers labeling some of $G_2$'s nodes, so that $G_1$ is colorable with 3 colors, iff a placement of mines on the remaining nodes of $G_2$ is possible, so that any node $v$ that is labeled $m$ has exactly $m$ neighboring nodes containing mines. \\

Let $G_1$ be any undirected graph. Now we show how to convert $G_1$ to $G_2$ along with numbers labeling some of $G_2$'s nodes. The graph $G_2$ contains gadgets for the nodes and edges in the graph $G_1$. The node gadget contains four nodes, where one of the four nodes has label 1 and it is connected to the other three nodes with an edge. The three unlabeled nodes represents the three colors of the palette.
\begin{center}
\includegraphics[scale=1.0]{Figures/Problem7.32a.pdf} \\
Node gadget
\end{center}

The edge gadget consists of a structure that connects the unlabeled nodes in two node gadgets, so that it is not possible to place mines in both of the two corresponding unlabeled nodes. 

\begin{center}
\includegraphics[scale=1.0]{Figures/Problem7.32b.pdf} \\
Edge gadget
\end{center}

Suppose that graph $G$ is colorable with 3 colors. Then,

\end{proof}

\end{document}