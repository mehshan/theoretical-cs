\documentclass[11pt]{article}


\usepackage{fullpage}
\usepackage{graphicx}
\usepackage{amsmath}
\usepackage{amssymb}
\usepackage{amsthm}
\usepackage{fancyvrb}

\newcommand{\myname}{Mehshan Mustafa}

\newenvironment{theorem}[2][Theorem]{\begin{trivlist}
\item[\hskip \labelsep {\bfseries #1}\hskip \labelsep {\bfseries #2.}]}{\end{trivlist}}
\newenvironment{lemma}[2][Lemma]{\begin{trivlist}
\item[\hskip \labelsep {\bfseries #1}\hskip \labelsep {\bfseries #2.}]}{\end{trivlist}}
\newenvironment{exercise}[2][Exercise]{\begin{trivlist}
\item[\hskip \labelsep {\bfseries #1}\hskip \labelsep {\bfseries #2.}]}{\end{trivlist}}
\newenvironment{problem}[2][Problem]{\begin{trivlist}
\item[\hskip \labelsep {\bfseries #1}\hskip \labelsep {\bfseries #2.}]}{\end{trivlist}}
\newenvironment{question}[2][Question]{\begin{trivlist}
\item[\hskip \labelsep {\bfseries #1}\hskip \labelsep {\bfseries #2.}]}{\end{trivlist}}
\newenvironment{corollary}[2][Corollary]{\begin{trivlist}
\item[\hskip \labelsep {\bfseries #1}\hskip \labelsep {\bfseries #2.}]}{\end{trivlist}}
\newenvironment{solution}{\begin{proof}[Solution]}{\end{proof}}
\newenvironment{idea}[2][Proof Idea.]{\textit{#1} #2}



\parindent0in
\pagestyle{plain}
\thispagestyle{plain}

\usepackage{csquotes}
\usepackage[shortlabels]{enumitem}

\newcommand{\dated}{\today}
\newcommand{\token}[1]{\langle \text{#1} \rangle}

\begin{document}

\textbf{Introduction to the Theory of
Computation}\hfill\textbf{\myname}\\[0.01in]
\textbf{Chapter 7: Time Complexity}\hfill\textbf{\dated}\\
\smallskip\hrule\bigskip

\begin{problem}{7.32}
This problem is inspired by the single-player game \textit{Minesweeper}, generalized to an arbitrary graph. Let $G$ be an undirected graph, where each node either contains a single, hidden \textit{mine} or is empty. The player chooses nodes, one by one. If the player chooses a node containing a mine, the player loses. If the player chooses an empty node, the player learns the number of neighboring nodes containing mines. (A neighboring node is one connected to the chosen node by an edge.) The player wins if and when all empty nodes have been so chosen.\\

In the \textit{mine consistency problem}, you are given a graph $G$ along with numbers labeling some of $G$'s nodes. You must determine whether a placement of mines on the remaining nodes is possible, so that any node $v$ that is labeled $m$ has exactly $m$ neighboring nodes containing mines. Formulate this problem as a language and show that it is NP-complete.
\end{problem}

\begin{proof}
Let $MINE-CONSISTENCY = \{\langle G \rangle \ | \ G \text{ is an undirected graph along with numbers} \\ \text{labeling some of } G\text{'s nodes, and a placement of mines on the remaining nodes is possible, so that} \\ \text{any node } v \text{ that is labeled } m \text{ has exactly } m \text{ neighboring nodes containing mines.}\}$.

May be do this....
Map 3SAT <G> to MINE-CONSISTENCY <G'>
 - Nodes in G are mapped to nodes in G' + additional unlabeled nodes
 - Nodes in G are assigned three different labels m1, m2, m3
 - Labels in G' gives the color of G

\end{proof}

\end{document}