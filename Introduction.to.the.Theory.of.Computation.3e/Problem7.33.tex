\documentclass[11pt]{article}


\usepackage{fullpage}
\usepackage{graphicx}
\usepackage{amsmath}
\usepackage{amssymb}
\usepackage{amsthm}
\usepackage{fancyvrb}

\newcommand{\myname}{Mehshan Mustafa}

\newenvironment{theorem}[2][Theorem]{\begin{trivlist}
\item[\hskip \labelsep {\bfseries #1}\hskip \labelsep {\bfseries #2.}]}{\end{trivlist}}
\newenvironment{lemma}[2][Lemma]{\begin{trivlist}
\item[\hskip \labelsep {\bfseries #1}\hskip \labelsep {\bfseries #2.}]}{\end{trivlist}}
\newenvironment{exercise}[2][Exercise]{\begin{trivlist}
\item[\hskip \labelsep {\bfseries #1}\hskip \labelsep {\bfseries #2.}]}{\end{trivlist}}
\newenvironment{problem}[2][Problem]{\begin{trivlist}
\item[\hskip \labelsep {\bfseries #1}\hskip \labelsep {\bfseries #2.}]}{\end{trivlist}}
\newenvironment{question}[2][Question]{\begin{trivlist}
\item[\hskip \labelsep {\bfseries #1}\hskip \labelsep {\bfseries #2.}]}{\end{trivlist}}
\newenvironment{corollary}[2][Corollary]{\begin{trivlist}
\item[\hskip \labelsep {\bfseries #1}\hskip \labelsep {\bfseries #2.}]}{\end{trivlist}}
\newenvironment{solution}{\begin{proof}[Solution]}{\end{proof}}
\newenvironment{idea}[2][Proof Idea.]{\textit{#1} #2}



\parindent0in
\pagestyle{plain}
\thispagestyle{plain}

\usepackage{csquotes}
\usepackage[shortlabels]{enumitem}

\newcommand{\dated}{\today}
\newcommand{\token}[1]{\langle \text{#1} \rangle}

\begin{document}

\textbf{Introduction to the Theory of
Computation}\hfill\textbf{\myname}\\[0.01in]
\textbf{Chapter 7: Time Complexity}\hfill\textbf{\dated}\\
\smallskip\hrule\bigskip

\begin{problem}{7.33}
In the following solitaire game, you are given an $m \times m$ board. On each of its $m^2$ positions lies either a blue stone, a red stone, or nothing at all. You play by removing stones from the board until each column contains only stones of a single color and each row contains at least one stone. You win if you achieve this objective. Winning may or may not be possible, depending upon the initial configuration. Let $SOLITAIRE = \{\langle G \rangle \ | \ G \text{ is a winnable game configuration}\}$. Prove that $SOLITAIRE$ is NP-complete.
\end{problem}

\begin{proof}
To show that $SOLITAIRE$ is NP-complete, we must show that it is in NP and that all NP-problems are polynomial time reducible to it. The first part is easy; a certificate is simply the set of positions that needs to be cleared. To prove the second part, we show that $3SAT$ is polynomial time reducible to $SOLITAIRE$. The reduction converts a Boolean formula $\phi$ in 3CNF into an $m \times m$ board $G$ having either a blue stone, a red stone, or nothing at all placed on each of its $m^2$ positions, so that $G$ is a winnable game configuration, iff $\phi$ is satisfiable. \\

Let $\phi$ be any Boolean formula in 3CNF containing $p$ clauses:
\[
\phi = (a_1 \vee b_1 \vee c_1) \ \wedge (a_2 \vee b_2 \vee c_2) \ \wedge \cdots \wedge (a_p \vee b_p \vee c_p).
\]
where each $a$, $b$ and $c$ is a literal $x_i$ or $\overline{x_i}$, and $x_1, x_2 \cdots x_n$ are the $n$ variables of $\phi$. Now we show how to convert $\phi$ to $G$. \\

The board $G$ contains a row for each clause in $\phi$ and 3 consecutive columns $3i - 2$, $3i - 1$ and $3i$ for each variable $x_i$ in $\phi$. For each clause $C_i$ in $\phi$, place stones in $G$ for each of the three literals in $C_i$ as follows: 

\begin{enumerate}
\item If $a_i$ is literal $x_j$, then place a red stone at row $i$ and column $3j - 2$. Otherwise, place a blue stone.
\item If $b_i$ is literal $x_j$, then place a red stone at row $i$ and column $3j - 1$. Otherwise, place a blue stone.
\item If $c_i$ is literal $x_j$, then place a red stone at row $i$ and column $3j$. Otherwise, place a blue stone.
\end{enumerate}

Next, we make sure that all three columns of a variable can only contain stones of a single color. This constraint is imposed by using the following configuration.

\begin{center}
\vspace{\baselineskip}
\includegraphics[scale=0.6]{Figures/Problem7.33a.pdf} \\
\vspace{\baselineskip}
\parbox{10cm}{The placement of stones in the bottom four rows ensure that all three columns can only contain stones of a single color.}
\vspace{\baselineskip}
\end{center}

The above configuration is repeatedly used in columns of every variable as shown in the following figure.

\begin{center}
\vspace{\baselineskip}
\includegraphics[scale=0.6]{Figures/Problem7.33b.pdf} \\
\vspace{\baselineskip}
Cells in the shaded area of the board are empty.
\vspace{\baselineskip}
\end{center}


\end{proof}

\end{document}