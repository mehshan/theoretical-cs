\documentclass[11pt]{article}

\usepackage{fullpage}
\usepackage{graphicx}
\usepackage{amsmath}
\usepackage{amssymb}
\usepackage{amsthm}
\usepackage{fancyvrb}

\parindent0in
\pagestyle{plain}
\thispagestyle{plain}

\newcommand{\myname}{Mehshan Mustafa}
\newcommand{\dated}{\today}

\newenvironment{theorem}[2][Theorem]{\begin{trivlist}
\item[\hskip \labelsep {\bfseries #1}\hskip \labelsep {\bfseries #2.}]}{\end{trivlist}}
\newenvironment{lemma}[2][Lemma]{\begin{trivlist}
\item[\hskip \labelsep {\bfseries #1}\hskip \labelsep {\bfseries #2.}]}{\end{trivlist}}
\newenvironment{exercise}[2][Exercise]{\begin{trivlist}
\item[\hskip \labelsep {\bfseries #1}\hskip \labelsep {\bfseries #2.}]}{\end{trivlist}}
\newenvironment{problem}[2][Problem]{\begin{trivlist}
\item[\hskip \labelsep {\bfseries #1}\hskip \labelsep {\bfseries #2.}]}{\end{trivlist}}
\newenvironment{question}[2][Question]{\begin{trivlist}
\item[\hskip \labelsep {\bfseries #1}\hskip \labelsep {\bfseries #2.}]}{\end{trivlist}}
\newenvironment{corollary}[2][Corollary]{\begin{trivlist}
\item[\hskip \labelsep {\bfseries #1}\hskip \labelsep {\bfseries #2.}]}{\end{trivlist}}
\newenvironment{solution}{\begin{proof}[Solution]}{\end{proof}}
\newenvironment{idea}[2][Proof Idea.]{\textit{#1} #2}

\begin{document}
\textbf{Introduction to the Theory of
Computation}\hfill\textbf{\myname}\\[0.01in]
\textbf{Chapter 1: Reqular Languages}\hfill\textbf{\dated}\\
\smallskip\hrule\bigskip

\begin{problem}{1.68}
In the traditional method for cutting a deck of playing cards, the deck is arbitrarily split two parts, which are exchanged before reassembling the deck. In a more complex cut, called Scarne’s cut, the deck is broken into three parts and the middle part in placed first in the reassembly. We’ll take Scarne’s cut as the inspiration for an operation on languages. For a language $A$, let $CUT(A) = \{yxz \ | \ xyz \in A\}$.
\end{problem}

\begin{problem}[Part]{a}
Exhibit a language $B$ for which $CUT(B) \neq CUT(CUT(B))$
\end{problem}

Let $\Sigma = \{a, b\}$ and $B = \{baab\}$. $CUT(B) = \{abab, aabb\}$, whereas $CUT(CUT(B)) = \{baab, aabb, abab\}$.

\begin{problem}[Part]{b}
Show that the class of regular languages is closed under $CUT$.
\end{problem}

\begin{proof}
Solution Replace this text with the details of your proof or solution.
\end{proof}
\end{document}