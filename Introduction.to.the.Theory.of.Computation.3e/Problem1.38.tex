\documentclass[11pt]{article}

\usepackage{fullpage}
\usepackage{graphicx}
\usepackage{amsmath}
\usepackage{amssymb}
\usepackage{amsthm}
\usepackage{fancyvrb}

\parindent0in
\pagestyle{plain}
\thispagestyle{plain}

\newcommand{\myname}{Mehshan Mustafa}
\newcommand{\dated}{December 3, 2024} % \today

\newenvironment{theorem}[2][Theorem]{\begin{trivlist}
\item[\hskip \labelsep {\bfseries #1}\hskip \labelsep {\bfseries #2.}]}{\end{trivlist}}
\newenvironment{lemma}[2][Lemma]{\begin{trivlist}
\item[\hskip \labelsep {\bfseries #1}\hskip \labelsep {\bfseries #2.}]}{\end{trivlist}}
\newenvironment{exercise}[2][Exercise]{\begin{trivlist}
\item[\hskip \labelsep {\bfseries #1}\hskip \labelsep {\bfseries #2.}]}{\end{trivlist}}
\newenvironment{problem}[2][Problem]{\begin{trivlist}
\item[\hskip \labelsep {\bfseries #1}\hskip \labelsep {\bfseries #2.}]}{\end{trivlist}}
\newenvironment{question}[2][Question]{\begin{trivlist}
\item[\hskip \labelsep {\bfseries #1}\hskip \labelsep {\bfseries #2.}]}{\end{trivlist}}
\newenvironment{corollary}[2][Corollary]{\begin{trivlist}
\item[\hskip \labelsep {\bfseries #1}\hskip \labelsep {\bfseries #2.}]}{\end{trivlist}}
\newenvironment{solution}{\begin{proof}[Solution]}{\end{proof}}
\newenvironment{idea}[2][Proof Idea.]{\textit{#1} #2}

\begin{document}
\textbf{Introduction to the Theory of
Computation}\hfill\textbf{\myname}\\[0.01in]
\textbf{Chapter 1: Regular Languages}\hfill\textbf{\dated}\\
\smallskip\hrule\bigskip

\begin{problem}{1.38}
An all-NFA $M$ is a 5-tuple $(Q, \Sigma, \delta, q_{0}, F)$ that accepts $x \in \Sigma^{*}$ if every possible state that $M$ could be in after reading input $x$ is a state from $F$. Note, in contrast, that an ordinary NFA accepts a string if some state among these possible states is an accept state. Prove that all-NFAs recognize the class of regular languages.
\end{problem}

\begin{proof}
To Do
\end{proof}
\end{document}