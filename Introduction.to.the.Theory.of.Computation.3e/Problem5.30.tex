\documentclass[11pt]{article}


\usepackage{fullpage}
\usepackage{graphicx}
\usepackage{amsmath}
\usepackage{amssymb}
\usepackage{amsthm}
\usepackage{fancyvrb}

\newcommand{\myname}{Mehshan Mustafa}

\newenvironment{theorem}[2][Theorem]{\begin{trivlist}
\item[\hskip \labelsep {\bfseries #1}\hskip \labelsep {\bfseries #2.}]}{\end{trivlist}}
\newenvironment{lemma}[2][Lemma]{\begin{trivlist}
\item[\hskip \labelsep {\bfseries #1}\hskip \labelsep {\bfseries #2.}]}{\end{trivlist}}
\newenvironment{exercise}[2][Exercise]{\begin{trivlist}
\item[\hskip \labelsep {\bfseries #1}\hskip \labelsep {\bfseries #2.}]}{\end{trivlist}}
\newenvironment{problem}[2][Problem]{\begin{trivlist}
\item[\hskip \labelsep {\bfseries #1}\hskip \labelsep {\bfseries #2.}]}{\end{trivlist}}
\newenvironment{question}[2][Question]{\begin{trivlist}
\item[\hskip \labelsep {\bfseries #1}\hskip \labelsep {\bfseries #2.}]}{\end{trivlist}}
\newenvironment{corollary}[2][Corollary]{\begin{trivlist}
\item[\hskip \labelsep {\bfseries #1}\hskip \labelsep {\bfseries #2.}]}{\end{trivlist}}
\newenvironment{solution}{\begin{proof}[Solution]}{\end{proof}}
\newenvironment{idea}[2][Proof Idea.]{\textit{#1} #2}



\parindent0in
\pagestyle{plain}
\thispagestyle{plain}

\usepackage{csquotes}
\usepackage[shortlabels]{enumitem}

\newcommand{\dated}{\today}
\newcommand{\token}[1]{\langle \text{#1} \rangle}

\begin{document}

\textbf{Introduction to the Theory of
Computation}\hfill\textbf{\myname}\\[0.01in]
\textbf{Chapter 5: Reducibility}\hfill\textbf{\dated}\\
\smallskip\hrule\bigskip

\begin{problem}{5.30}
Use Rice’s theorem, which appears in Problem 5.28, to prove the undecidability of each of the following languages.
\end{problem}

\begin{problem}[Part]{b}
$\{\langle M \rangle \ | \ M \text{ is a } TM \text{ and } 1011 \in L(M)\}$.
\end{problem}

\begin{proof}
Let $P = \{\langle M \rangle \ | \ M \text{ is a } TM \text{ and } 1011 \in L(M)\}$. $P$ is a language of \textbf{\textit{TM}} descriptions. It satisfies the two conditions of Rice’s theorem. First, it is nontrivial because languages some \textbf{\textit{TM}}s contain 1011 and others do not. Second, it depends only on the language. If two \textbf{\textit{TM}}s recognize the same language, either both have descriptions in $P$ or neither do. Consequently, Rice’s theorem implies that $P$ is undecidable.
\end{proof}

\begin{problem}[Part]{c}
$ALL_{TM} = \{\langle M \rangle \ | \ M \text{ is a } TM \text{ and } L(M ) = \Sigma^{*}\}$.
\end{problem}

\begin{proof}
$ALL_{TM}$ is a language of \textbf{\textit{TM}} descriptions. It satisfies the two conditions of Rice’s theorem. First, it is nontrivial because languages some \textbf{\textit{TM}}s contain all possible strings and others do not. Second, it depends only on the language. If two \textbf{\textit{TM}}s recognize the same language, either both have descriptions in $ALL_{TM}$ or neither do. Consequently, Rice’s theorem implies that $ALL_{TM}$ is undecidable.
\end{proof}

\end{document}