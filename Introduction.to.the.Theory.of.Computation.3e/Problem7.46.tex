\documentclass[11pt]{article}


\usepackage{fullpage}
\usepackage{graphicx}
\usepackage{amsmath}
\usepackage{amssymb}
\usepackage{amsthm}
\usepackage{fancyvrb}

\newcommand{\myname}{Mehshan Mustafa}

\newenvironment{theorem}[2][Theorem]{\begin{trivlist}
\item[\hskip \labelsep {\bfseries #1}\hskip \labelsep {\bfseries #2.}]}{\end{trivlist}}
\newenvironment{lemma}[2][Lemma]{\begin{trivlist}
\item[\hskip \labelsep {\bfseries #1}\hskip \labelsep {\bfseries #2.}]}{\end{trivlist}}
\newenvironment{exercise}[2][Exercise]{\begin{trivlist}
\item[\hskip \labelsep {\bfseries #1}\hskip \labelsep {\bfseries #2.}]}{\end{trivlist}}
\newenvironment{problem}[2][Problem]{\begin{trivlist}
\item[\hskip \labelsep {\bfseries #1}\hskip \labelsep {\bfseries #2.}]}{\end{trivlist}}
\newenvironment{question}[2][Question]{\begin{trivlist}
\item[\hskip \labelsep {\bfseries #1}\hskip \labelsep {\bfseries #2.}]}{\end{trivlist}}
\newenvironment{corollary}[2][Corollary]{\begin{trivlist}
\item[\hskip \labelsep {\bfseries #1}\hskip \labelsep {\bfseries #2.}]}{\end{trivlist}}
\newenvironment{solution}{\begin{proof}[Solution]}{\end{proof}}
\newenvironment{idea}[2][Proof Idea.]{\textit{#1} #2}



\parindent0in
\pagestyle{plain}
\thispagestyle{plain}

\usepackage{csquotes}
\usepackage[shortlabels]{enumitem}

\newcommand{\dated}{\today}
\newcommand{\token}[1]{\langle \text{#1} \rangle}

\begin{document}

\textbf{Introduction to the Theory of
Computation}\hfill\textbf{\myname}\\[0.01in]
\textbf{Chapter 7: Time Complexity}\hfill\textbf{\dated}\\
\smallskip\hrule\bigskip

\begin{problem}{7.46}
Say that two Boolean formulas are \textit{\textbf{equivalent}} if they have the same set of variables and are true on the same set of assignments to those variables (i.e., they describe the same Boolean function). A Boolean formula is \textit{\textbf{minimal}} if no shorter Boolean formula is equivalent to it. Let $MIN-FORMULA$ be the collection of minimal Boolean formulas. Show that if P = NP, then $MIN-FORMULA \in P$. 
\end{problem}

\begin{proof}
The P = NP assumption implies that $SAT$ is in $P$, so testing satisfiability is solvable in polynomial time. To show that if P = NP, then $MIN-FORMULA \in P$, we give a polynomial time algorithm $M$. \\

$M =$ \textquotedblleft On input $\langle \phi \rangle$, where $\phi$ is a Boolean formula:
\begin{enumerate}
\item Repeat for every literal $a$ in $\phi$:
\item \hspace*{0.5cm} Construct a new Boolean formula $\phi_t$ by replacing $a$ with $true$ in $\phi$.
\item \hspace*{0.5cm} Construct a new Boolean formula $\phi_f$ by replacing $a$ with $false$ in $\phi$.
\item \hspace*{0.5cm} Test satisfiablity of $\phi_t$ and $\phi_f$.
\item \hspace*{0.5cm} If satisfiability of $\phi_t$ is same as $\phi_f$, then \textit{reject}.
\item \textit{Accept}.\textquotedblright
\end{enumerate}

\end{proof}

\end{document}