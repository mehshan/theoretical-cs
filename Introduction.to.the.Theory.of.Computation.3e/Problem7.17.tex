\documentclass[11pt]{article}


\usepackage{fullpage}
\usepackage{graphicx}
\usepackage{amsmath}
\usepackage{amssymb}
\usepackage{amsthm}
\usepackage{fancyvrb}

\newcommand{\myname}{Mehshan Mustafa}

\newenvironment{theorem}[2][Theorem]{\begin{trivlist}
\item[\hskip \labelsep {\bfseries #1}\hskip \labelsep {\bfseries #2.}]}{\end{trivlist}}
\newenvironment{lemma}[2][Lemma]{\begin{trivlist}
\item[\hskip \labelsep {\bfseries #1}\hskip \labelsep {\bfseries #2.}]}{\end{trivlist}}
\newenvironment{exercise}[2][Exercise]{\begin{trivlist}
\item[\hskip \labelsep {\bfseries #1}\hskip \labelsep {\bfseries #2.}]}{\end{trivlist}}
\newenvironment{problem}[2][Problem]{\begin{trivlist}
\item[\hskip \labelsep {\bfseries #1}\hskip \labelsep {\bfseries #2.}]}{\end{trivlist}}
\newenvironment{question}[2][Question]{\begin{trivlist}
\item[\hskip \labelsep {\bfseries #1}\hskip \labelsep {\bfseries #2.}]}{\end{trivlist}}
\newenvironment{corollary}[2][Corollary]{\begin{trivlist}
\item[\hskip \labelsep {\bfseries #1}\hskip \labelsep {\bfseries #2.}]}{\end{trivlist}}
\newenvironment{solution}{\begin{proof}[Solution]}{\end{proof}}
\newenvironment{idea}[2][Proof Idea.]{\textit{#1} #2}



\parindent0in
\pagestyle{plain}
\thispagestyle{plain}

\usepackage{csquotes}
\usepackage[shortlabels]{enumitem}

\newcommand{\dated}{\today}
\newcommand{\token}[1]{\langle \text{#1} \rangle}

\begin{document}

\textbf{Introduction to the Theory of
Computation}\hfill\textbf{\myname}\\[0.01in]
\textbf{Chapter 7: Time Complexity}\hfill\textbf{\dated}\\
\smallskip\hrule\bigskip

\begin{problem}{7.17}
Let $UNARY\text{-}SSUM$ be the subset sum problem in which all numbers are represented in unary. Why does the NP-completeness proof for $SUBSET\text{-}SUM$ fail to show $UNARY\text{-}SSUM$ is NP-complete? Show that $UNARY\text{-}SSUM \in P$.
\end{problem} 

\begin{problem}[Part]{a}
Why does the NP-completeness proof for $SUBSET\text{-}SUM$ fail to show $UNARY\text{-}SSUM$ is NP-complete?
\end{problem}

\begin{problem}[Part]{b}
Show that $UNARY\text{-}SSUM \in P$.
\end{problem}

\begin{idea}The $UNARY\text{-}SSUM$ problem involving $S = \{x_1, \cdots, x_k\}$, and $t$ reduces to testing if $t \in S^{*}$. For example, let $S_1 = \{\varepsilon, 11, 1111\}$, and $t_1 = 111111$. Clearly, $t_1 \in S_1^{*}$, therefore $\langle S_1, t_1 \rangle \in UNARY\text{-}SSUM$. $S$ is finite, therefore membership in $S$, $MEMBER_S = \{t \ | \ t \in S \}$ can be decided in polynomial time. $MEMBER_S \in P$, and $P$ is closed under the star operation \footnote{See solution to Problem 7.15.}, therefore $UNARY\text{-}SSUM \in P$.
\end{idea}
 
\begin{proof}
Let
\begin{align*} 
UNARY\text{-}SSUM = \{\langle S, t \rangle \ | \ S = \{x_1, \cdots, x_k\}, \text{ and for some } \\
\{y_1, \cdots , y_l\} \subseteq \{x_1, \cdots, x_k\}, \text{ we have } \Sigma y_i = t, \\
\text{ and } x_i, \ y_j \text{ and } t \text{ are represented in unary}\}.
\end{align*}
To show $UNARY\text{-}SSUM \in P$, we give a polynomial time reduction from $UNARY\text{-}SSUM$ to $MEMBER_S$ as described above. \\

$F =$ \textquotedblleft On input $\langle S, t \rangle$, where $S = \{x_1, \cdots, x_k\}$, each $x_i$ and $t$ is a non-negative number represented in unary:
\begin{enumerate}
\item If $t = \varepsilon$ and $\varepsilon \notin S$, then output $\langle \{11\}, 1  \rangle$.
\item Output $\langle S, t \rangle$.\textquotedblright
\end{enumerate}
\end{proof}

\end{document}