\documentclass[11pt]{article}

\usepackage{fullpage}
\usepackage{graphicx}
\usepackage{amsmath}
\usepackage{amssymb}
\usepackage{amsthm}
\usepackage{fancyvrb}

\parindent0in
\pagestyle{plain}
\thispagestyle{plain}

\newcommand{\myname}{Mehshan Mustafa}
\newcommand{\dated}{\today}

\newenvironment{theorem}[2][Theorem]{\begin{trivlist}
\item[\hskip \labelsep {\bfseries #1}\hskip \labelsep {\bfseries #2.}]}{\end{trivlist}}
\newenvironment{lemma}[2][Lemma]{\begin{trivlist}
\item[\hskip \labelsep {\bfseries #1}\hskip \labelsep {\bfseries #2.}]}{\end{trivlist}}
\newenvironment{exercise}[2][Exercise]{\begin{trivlist}
\item[\hskip \labelsep {\bfseries #1}\hskip \labelsep {\bfseries #2.}]}{\end{trivlist}}
\newenvironment{problem}[2][Problem]{\begin{trivlist}
\item[\hskip \labelsep {\bfseries #1}\hskip \labelsep {\bfseries #2.}]}{\end{trivlist}}
\newenvironment{question}[2][Question]{\begin{trivlist}
\item[\hskip \labelsep {\bfseries #1}\hskip \labelsep {\bfseries #2.}]}{\end{trivlist}}
\newenvironment{corollary}[2][Corollary]{\begin{trivlist}
\item[\hskip \labelsep {\bfseries #1}\hskip \labelsep {\bfseries #2.}]}{\end{trivlist}}
\newenvironment{solution}{\begin{proof}[Solution]}{\end{proof}}
\newenvironment{idea}[2][Proof Idea.]{\textit{#1} #2}

\begin{document}

\textbf{Introduction to the Theory of
Computation}\hfill\textbf{\myname}\\[0.01in]
\textbf{Chapter 1: Reqular Languages}\hfill\textbf{\dated}\\
\smallskip\hrule\bigskip

\begin{problem}{1.54}
Consider the language $F = \{a^{i}b^{j}c^{k} \ | \ i, \ j, \ k \geq 0 \ and \ if \ i = 1 \ then \ j = k\}$.
\end{problem}

\begin{problem}[Part]{a}
Show that $F$ is not regular.
\end{problem}
\begin{proof}
To show that $F$ is not regular, we use \textbf{Myhill–Nerode theorem}, and show that the \textbf{index of \textit{L}}\footnote{Problem 1.52 \textbf{Myhill–Nerode theorem}.} is infinite. Define
\[ X = \{w \ | \ w = ab^{i}, \ where \ i \geq 0 \}. \]
The set $X$ is infinite, and no two strings in $X$ have the same number of b's. Take any two strings $w_{a}, w_{b} \in X$, let $k$ be the number of b's in $w_{a}$, and let $z = c^{k}$, then $w_{a}z \in F$ and $w_{b}z \notin F$. Thus $X$ is pairwise distinguishable by $F$, and $X$ is the index of $L$. As $X$ is infinite, therefore $F$ is not a regular language.
\end{proof}

\begin{problem}[Part]{b}
Show that $F$ acts like a regular language in the pumping lemma. In other
words, give a pumping length $p$ and demonstrate that $F$ satisfies the three conditions of the pumping lemma for this value of $p$.
\end{problem}
Let the pumping length $p=2$. For any string $s \in F$ with a length of at least $2$, there can be four cases regarding the number of a's.
\begin{enumerate}
\item No a's. \\
Split $s = xyz$, where $x = \epsilon$, $y$ is the first symbol and $z$ is the rest.
\item Exactly one a, and an equal number of b's and c's. \\
Split $s = xyz$, where $x = \epsilon$, $y = a$ and $z$ is the rest.
\item Exactly two a's. \\
Split $s = xyz$, where $x = \epsilon$, $y=aa$ and $z$ is the rest.
\item More than two a's. \\
Split $s = xyz$, where $x = \epsilon$, $y=a$ and $z$ is the rest.
\end{enumerate}
In each case, the string $s$ can be split into three pieces, $s = xyz$, satisfying the three conditions of the pumping lemma.

\begin{problem}[Part]{c}
Explain why parts (a) and (b) do not contradict the pumping lemma.
\end{problem}
The pumping lemma is an implication, where the antecedent is that a given language is regular, and the consequent is that all large enough strings in the given language can be split and pumped. In case of $F$, the antecedent is false, so the consequent may or may not be true.

\end{document}