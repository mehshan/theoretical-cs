\documentclass[11pt]{article}


\usepackage{fullpage}
\usepackage{graphicx}
\usepackage{amsmath}
\usepackage{amssymb}
\usepackage{amsthm}
\usepackage{fancyvrb}

\newcommand{\myname}{Mehshan Mustafa}

\newenvironment{theorem}[2][Theorem]{\begin{trivlist}
\item[\hskip \labelsep {\bfseries #1}\hskip \labelsep {\bfseries #2.}]}{\end{trivlist}}
\newenvironment{lemma}[2][Lemma]{\begin{trivlist}
\item[\hskip \labelsep {\bfseries #1}\hskip \labelsep {\bfseries #2.}]}{\end{trivlist}}
\newenvironment{exercise}[2][Exercise]{\begin{trivlist}
\item[\hskip \labelsep {\bfseries #1}\hskip \labelsep {\bfseries #2.}]}{\end{trivlist}}
\newenvironment{problem}[2][Problem]{\begin{trivlist}
\item[\hskip \labelsep {\bfseries #1}\hskip \labelsep {\bfseries #2.}]}{\end{trivlist}}
\newenvironment{question}[2][Question]{\begin{trivlist}
\item[\hskip \labelsep {\bfseries #1}\hskip \labelsep {\bfseries #2.}]}{\end{trivlist}}
\newenvironment{corollary}[2][Corollary]{\begin{trivlist}
\item[\hskip \labelsep {\bfseries #1}\hskip \labelsep {\bfseries #2.}]}{\end{trivlist}}
\newenvironment{solution}{\begin{proof}[Solution]}{\end{proof}}
\newenvironment{idea}[2][Proof Idea.]{\textit{#1} #2}



\parindent0in
\pagestyle{plain}
\thispagestyle{plain}

\usepackage{csquotes}
\usepackage[shortlabels]{enumitem}

\newcommand{\dated}{\today}
\newcommand{\token}[1]{\langle \text{#1} \rangle}

\begin{document}

\textbf{Introduction to the Theory of
Computation}\hfill\textbf{\myname}\\[0.01in]
\textbf{Chapter 5: Reducibility}\hfill\textbf{\dated}\\
\smallskip\hrule\bigskip

\begin{problem}{5.13}
A \textbf{\textit{useless state}} in a Turing machine is one that is never entered on any input string. Consider the problem of determining whether a Turing machine has any useless states. Formulate this problem as a language and show that it is undecidable.
\end{problem}

\begin{proof}
Let $T = \{\langle M \rangle \ | \ M \text{ is a \textbf{TM}, which has a useless state}\}$. Show that $A_{TM}$\ reduces to $T$, where $A_{TM} = \{\langle M, w \rangle \ | \ M \text{ is a \textbf{TM} and } M \text{ accepts } w\}$. Assume for the sake of contradiction that \textbf{TM} $R$ decides $T$. Then construct a \textbf{TM} $S$ that uses $R$ to decide $A_{TM}$. The idea is to modify the input \textbf{TM} $M$, so that the modified \textbf{TM} $M_w$ does not have any useless states if $M$ accepts $w$. We can do this by modifying $M$ as follows:
\begin{enumerate}
\item Add special new state $q_u$.
\item Before processing the input, transition through all non-start states except $q_u$.
\item Simulate $M$ on $w$.
\item If the simulation shows that $M$ accepts $w$, transition to $q_u$.
\end{enumerate}

$S =$ \textquotedblleft On input $\langle M, w \rangle$, where $M$ is a \textbf{TM} and $w$ is a string:
\begin{enumerate}
\item Transform $M$ to new \textbf{TM} $M_w$ as discussed above.
\item Run $R$ on $\langle M_w \rangle$.
\item If $R$ rejects, $M$ accepts $w$, so \textit{accept}. Otherwise, \textit{reject}.\textquotedblright
\end{enumerate}
Thus, if \textbf{TM} $R$ exists, we can decide $A_{TM}$, but we know that $A_{TM}$ is undecidable\footnote{Theorem 4.11 $A_{TM}$ is undecidable.}. By virtue of this contradiction, we can conclude that $R$ does not exist. Therefore, $T$ is undecidable.
\end{proof}
\end{document}