\documentclass[11pt]{article}


\usepackage{fullpage}
\usepackage{graphicx}
\usepackage{amsmath}
\usepackage{amssymb}
\usepackage{amsthm}
\usepackage{fancyvrb}

\newcommand{\myname}{Mehshan Mustafa}

\newenvironment{theorem}[2][Theorem]{\begin{trivlist}
\item[\hskip \labelsep {\bfseries #1}\hskip \labelsep {\bfseries #2.}]}{\end{trivlist}}
\newenvironment{lemma}[2][Lemma]{\begin{trivlist}
\item[\hskip \labelsep {\bfseries #1}\hskip \labelsep {\bfseries #2.}]}{\end{trivlist}}
\newenvironment{exercise}[2][Exercise]{\begin{trivlist}
\item[\hskip \labelsep {\bfseries #1}\hskip \labelsep {\bfseries #2.}]}{\end{trivlist}}
\newenvironment{problem}[2][Problem]{\begin{trivlist}
\item[\hskip \labelsep {\bfseries #1}\hskip \labelsep {\bfseries #2.}]}{\end{trivlist}}
\newenvironment{question}[2][Question]{\begin{trivlist}
\item[\hskip \labelsep {\bfseries #1}\hskip \labelsep {\bfseries #2.}]}{\end{trivlist}}
\newenvironment{corollary}[2][Corollary]{\begin{trivlist}
\item[\hskip \labelsep {\bfseries #1}\hskip \labelsep {\bfseries #2.}]}{\end{trivlist}}
\newenvironment{solution}{\begin{proof}[Solution]}{\end{proof}}
\newenvironment{idea}[2][Proof Idea.]{\textit{#1} #2}



\parindent0in
\pagestyle{plain}
\thispagestyle{plain}

\usepackage{csquotes}
\usepackage[shortlabels]{enumitem}

\newcommand{\dated}{\today}
\newcommand{\token}[1]{\langle \text{#1} \rangle}

\begin{document}

\textbf{Introduction to the Theory of
Computation}\hfill\textbf{\myname}\\[0.01in]
\textbf{Chapter 7: Time Complexity}\hfill\textbf{\dated}\\
\smallskip\hrule\bigskip

\begin{problem}{7.45}
Modify the algorithm for context-free language recognition in the proof of Theorem
7.16 to give a polynomial time algorithm that produces a parse tree for a
string, given the string and a CFG, if that grammar generates the string.
\end{problem}

\begin{solution}
Let $G$ be a CFG in Chomsky normal form generating the CFL $L$. Assume that $S$ is the start variable. The following algorithm $D'$ produces a parse tree for a
string in polynomial time, given the string and a CFG. \\

$D' =$ \textquotedblleft On input $w = w_1 \cdots w_n$:
\begin{enumerate}
\item For $w = \varepsilon$, if $S \rightarrow \varepsilon$ is a rule, \textit{accept}; else, \textit{reject}. \hfill [[$w = \varepsilon$ case]]
\item For $i = 1$ to $n$: \hfill [[ examine each substring of length 1 ]]
\item \hspace*{0.5cm} For each variable $A$:
\item \hspace*{1.0cm} Test whether $A \rightarrow b$ is a rule, where $b = w_i$.
\item \hspace*{1.0cm} If so, place $A$ in $table(i, i)$.
\item For $l = 2$ to $n$: \hfill [[ l is the length of the substring ]]
\item \hspace*{0.5cm} For $i = 1$ to $n - l + 1$: \hfill [[ i is the start position of the substring ]]
\item \hspace*{1.0cm} Let $j = i + l - 1$. \hfill [[ j is the end position of the substring ]]
\item \hspace*{1.0cm} For $k = i$ to $j - 1$: \hfill [[ $k$ is the split position ]]
\item \hspace*{1.5cm} For each rule $A \rightarrow BC$:
\item \hspace*{2.0cm} If $table(i, k)$ contains $B$ and $table(k + 1, j)$ contains $C$, \\ \hspace*{2.0cm} put $A$ in $table(i, j)$.
\item If $S$ is in $table(1, n),$ then:
\item 
\item Otherwise, output $\varepsilon$.\textquotedblright
\end{enumerate}
\end{solution}

\end{document}